\mainmatter
\chapter{Introduzione}
Il riciclaggio di denaro è una particolare tipologia di frode molto frequente all'interno degli istituiti bancari, per questo motivo studiare tecniche e approcci che ne permettano l'individuazione e la conseguente segnalazione all'autorità è di fondamentale importanza.
Il progetto di stage si è focalizzato su questo tema, andando a studiare un approccio che attraverso l'analisi delle transazioni di una banca portasse all'individuazione di schemi e comportamenti fraudolenti tra i clienti.

Il comportamento di ogni cliente viene analizzato per andare ad individuare delle anomalie nel suo comportamento durante il tempo, ad esempio se le sue abitudini di spesa sono costanti o ha dei picchi anomali in determinati periodo di tempo non coincidenti con festività o periodi di tassazione. 
Inoltre, ognuno dei clienti non viene analizzato solo singolarmente ma bensì anche in relazione agli altri utenti presenti nella banca, stabilendo se il suo comportamento è anomalo e quindi va a discostarsi significativamente dagli altri oppure rimane coerente con la popolazione.

Per effettuare queste analisi l'approccio che viene proposto si basa su delle tecniche di Machine Learning supervisionato e non supervisionato, che combinate portano alla delineazione di comportamenti fraudolenti e alla conseguente segnalazione di clienti anomali. 